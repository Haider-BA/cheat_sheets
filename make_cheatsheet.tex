% Define command to break line inside a table cell
\newcommand{\specialcell}[2][c]{%
\begin{tabular}[#1]{@{}c@{}}#2\end{tabular}}

\documentclass[letterpaper]{cheatsheet} % Use US Letter paper, change to a4paper for A4 

\begin{document}
%
%----------------------------------------------------------------------------------------
%	TITLE SECTION
%----------------------------------------------------------------------------------------
%
%\lastupdated % Print the Last Updated text at the top right
%\copyrights
\titlesection{GNU Make}{Cheat Sheet}{ % Your name
}
%
\begin{minipage}[t]{0.66\textwidth}
\section{Rules}
Short rules can be written as below\\
\texttt{target(s): [prerequisites] [; shell-command(s)]}\\
\texttt{target(s): [prerequisites]}\\
\texttt{ \hphantom{\texttt{target(s):} }[shell-command]}\\
\texttt{\%.class: \%.java; javac \$<} \\
A target pattern is composed of a ``\texttt{\%}''  between a prefix and a suffix, either or both of which may be empty.
\sectionspace
%
\section{Macros/Variables} 
%
Values assigned to variables with \texttt{=} are expanded when used (like a reference) while values assigned with \texttt{:=} are expanded at declaration time. 
\texttt{\$(myvar)} returns the stored value and \texttt{\$(call myvar)} executes the value (like a function).\\ \\
\begin{tabular}{@{}>{\ttfamily}p{0.3\textwidth}@{}p{0.7\textwidth}@{}}
\$@	& The name of the target.\\
\$\%	& The target member name, when the target is an archive member.\\
\$< & The name of the first (or only) prerequisite.\\
\$? & Space separated list of all prerequisites newer than than the target.\\
\$\^{} | \$+ & Space separated list of all prerequisites. \texttt{\$\^{}} omits duplicate prerequisites, while \texttt{\$+} retains them and preserves their order.\\
\$* & The stem with which an implicit rule matches.\\
\$(\placeholder D) | \$(\placeholder F) & The path and the file name of a macro. For instance \texttt{\$(@D)} returns the directory-part of \texttt{\$@}.\\
\end{tabular}
\sectionspace
\end{minipage}%
\hfill%
\begin{minipage}[t]{0.3\textwidth}
\section{Built-in target names}
\begin{tabular}{@{}>{\ttfamily}p{0.3\textwidth}@{}p{0.7\textwidth}@{}}
.PHONY &  prerequisites of a \texttt{.PHONY} target are never considered to be up to date.\\
.INTERMEDIATE & {%
	\newlength\xw
	\settowidth\xw{\texttt{.INTERMEDIATE}}
	\addtolength\xw{-0.3\textwidth}
	\hspace{\xw}}
	the targets which \texttt{.INTERMEDIATE} depends on are treated as intermediate files and they are deleted once no longer needed for a rule.\\
.SECONDARY & the targets which \texttt{.SECONDARY} depends on are treated as intermediate files but they are not automatically deleted.\\
.IGNORE & errors encountered while executing recipes for the prerequisites of \texttt{.IGNORE} are ignored.\\
.EXPORT\_ALL\_VARIABLES & {%
	\settowidth\xw{\texttt{.EXPORT\_ALL\_VARIABLES}}
	\addtolength\xw{-0.3\textwidth}
	\hspace{\xw}} 
	allows variables exported in parent Makefile to be available to rules in child processes. It uses no prerequisites.\\
\end{tabular}
\sectionspace
\end{minipage}

\sectionspace
\section{Includes}
\noindent For large projects use several makefiles and use the command \texttt{include} to call them within your 'master' makefile. Use \texttt{-include} \args{Makefile(s)} if you don't want \texttt{make} to abort when the included Makefile's missing. The minus sign generally forces \texttt{make} to ignore errors.
%
\sectionspace
\section{Functions}
\noindent
%\descript{Text formatting}
\begin{tabular}{@{}>{\ttfamily}p{0.38\textwidth}@{}p{0.62\textwidth}@{}}
\$(subst \args{from},\args{to},\args{text})	& Replaces each occurrence of \args{from} in \args{text} by \args{to}\\
\$(patsubst \args{pattern},\args{replacement},\args{text}) & Replaces words matching \args{pattern} with \args{replacement} in \args{text}.\\
\$(strip \args{string}) & 	Removes excess whitespace characters from \args{string}.\\
\$(findstring \args{find},\args{in})	& Search \args{in} for an occurrence of \args{find}.\\
\$(filter \args{pattern\_1} \args{pattern\_2\ldots},\args{text})	& Selects words in \args{text} that match one of the \args{pattern} words.\\
\$(filter-out \args{pattern\_1} \args{pattern\_2\ldots},\args{text}) & Selects words in text that do not match any of the pattern words.\\
\$(sort \args{list})	& Sorts the words in \args{list} lexicographically, removing duplicates.\\
\$(dir \args{names\ldots})	& Extracts the directory-part of each file name in \args{names}.\\
\$(notdir \args{names\ldots}) &	Extracts the non-directory part of each file name in \args{names}.\\
\$(realpath \args{names\ldots}) & Returns an absolute name (does not contain ``. or .. or symlinks'' for each file name in \args{names}.\\
\$(suffix \args{names\ldots})	& Extracts the suffix (everything starting with the last period) of each file name in \args{names}.\\
\$(basename \args{names\ldots}) & Extracts all but the suffix of each file name in \args{names}. \\
\$(addsuffix \args{suffix},\args{names\ldots}) & 	Appends \args{suffix} to each word in \args{names}.\\
\$(addprefix \args{prefix},\args{names\ldots}) & Prepends \args{prefix} to each word in \args{names}.\\
\$(join \args{list\_1},\args{list\_2})	& Join two parallel lists of words.\\
\$(word \args{n},\args{text}) & Extracts the \args{$n^{th}$} word of \args{text}.\\
\$(words \args{text})	 & Counts the number of words in \args{text}.\\
\$(wordlist \args{i},\args{j},\args{text}) & Returns the list of words in \args{text} from \args{i} to \args{j}.\\
\$(firstword \args{names\ldots}) & Extracts the first word in \args{names}.\\
\$(wildcard \args{pattern\ldots}) & Returns the file names matching (a shell file name) \args{pattern} (not a ``\%'' pattern).\\
\$(error \args{text\ldots}) & When the function is evaluated a fatal error with the message \args{text} is generated.\\
\$(warning \args{text\ldots}) & When the function is evaluated a warning with the message \args{text} is generated.\\
\$(shell \args{command}) & Execute a shell command and return its output.\\
\$(origin \args{variable}) & Describes where \args{variable} came from. Do not use ``\texttt{\$}''  or parentheses around \args{variable} unless you want the name not to be constant and provide a variable reference.\\
\$(foreach \args{var},\args{words},\args{text}) & Evaluate \args{text} with \args{var} bound to each word in \args{words}, and concatenate the results.\\
\$(call \args{var},\args{param},\args{param},\args{\ldots}) & Evaluate \args{var} replacing any references to \texttt{\$(1)},\texttt{\$(2)} with the first, second, etc. \args{param} values.\\
%
\end{tabular}
%
\noindent\let\thefootnote\relax\footnote{Inspired by \url{http://www.schacherer.de/frank/technology/tools/make.html} and \url{https://www.gnu.org/software/make/manual/} \\Created by \href{https://github.com/mxenoph}{\bf Maria Xenophontos} after many valuable comments from \href{https://github.com/klmr}{\bf Konrad Rudolph}}
%
\end{document}
